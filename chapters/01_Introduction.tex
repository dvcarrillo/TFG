\chapter{Introduction}

\say{I am a HAL 9000 computer. I became operational at the H.A.L. plant in Urbana, Illinois on the 12th of January 1992. My instructor 
was Mr. Langley, and he taught me to sing a song.}

These words were spoken by HAL 9000, the artificial general intelligence depicted in the movie \textit{2001: A Space Odyssey} 
by Stanley Kubrick, published back in 1968. In this film, HAL 9000 is in charge of controlling the systems of the 
\textit{Discovery One} spacecraft and interacting with the ship's astronaut crew. The abilities of this computer were impressive: it 
was capable of speech recognition, facial recognition, natural language processing, automated reasoning and many other features 
characteristic of the most complete artificial intelligence ever created. And on top  of that, it was also capable of doing tasks 
that are now known as home automation.

Of course, in 1968 the field of Artificial Intelligence was only taking its first steps, and these features were only a dream in many
people's minds. Nevertheless, \textit{2001: A Space Odyssey} contributed greatly to the popularization of these technologies among 
the general public.

Today, home automation and voice assistance are experiencing one of their most popular moments, thanks to the lower cost of components 
and the incredible development of Artificial Intelligence and Internet of Things by leading companies. And most importantly, these 
long-awaited technologies are finally within everyone's reach.

\section{Incentive}
The interest from companies about home automation and voice assistance has been growing in this decade. Currently, we can find
solutions from technology companies that combine a virtual assistant with a home automation system, and these are exactly the most 
popular ones. 

On the other hand, companies that have classically made home appliances and lightning systems, are now entering the smart home market.
The range of \textit{smart devices} is enormous at the moment, and many users may feel lost when looking for a solution for their homes.
This is one of the problems that I identified, but not the only one.

Another big problem is that home automation products tend to work only with other devices from the same maker. For example, Philips
lightning systems require a Philips bridge and a Philips mobile application in order to work. But if the user has lights from different 
makers, he will probably need to install more bridges and more applications in his mobile phone. However, all bridges usually do the same
job: receiving commands via WiFi or cable and sending them to the domotic devices via Zigbee or Z-Wave, for example (both are popular
communication protocols in domotics). Makers are not moving towards unification, but to differentiation.

Luckily, there are some systems that can unify a bit a home automation system composed by devices from different makers. For example,
Apple HomeKit or Amazon Alexa. However, these products are usually expensive and their customization is very limited. They also fall short 
of availability, as these previous devices are not yet available in Spain, nor in a large number of countries.

\section{Objectives}
From the previous incentive, we can see the need for an affordable and customizable home automation system that can group devices
from different manufacturers, and that offers as many facilities as the systems mentioned before. For example, an user-friendly 
interface and a virtual assistant.

\subsection{General Objectives}
\begin{enumerate}
	\item Design a domotic system that groups effectively home automation devices from different makers.
	\item Include extra facilities, such as automation or management from a mobile application.
	\item Make the system modular, extensible, safe and fully customizable by the user.
	\item Make the system accessible and adaptable, that is, having the ability to use it with an attached screen, an external 
	screen or only using the voice.
	\item Make the common processes (adding devices, configuring them, making automation rules) seamless and easy.
\end{enumerate}

\subsection{Specific Objectives}
\begin{enumerate}
	\item Integrate a home automation system in a embedded system, like a Raspberry Pi.
	\item Integrate a voice assistant in the same embedded system as the home automation system.
	\item Explore current home automation systems and voice assistants, focusing on open-source solutions.
	\item Explore automation possibilities and implement an automation service in the domotic system.
	\item Explore options for managing the system from a mobile application.
	\item Explore options for providing global access to the system and implement one.
	\item Explore safety and privacy concerns related to the home automation system.
	\item Provide an adaptive and responsive user interface, usable on touch and non-touch screens.
	\item Connect the virtual assistant to openHAB, so it can manage the devices present in the system.
	\item Test domotic devices in the final system and present an usable solution.
\end{enumerate}

\section{Structure of the Work}
This work is structured in seven chapters. The objective is to introduce first all the results of my research, that is, the general
and specific concepts and the most important products related to this project to provide a knowledge base in order to better 
understand the development of the resultant project.

Chapter 2 introduces the home automation technology. In this chapter, I explore the different concepts of home automation,
its main features and its history. I also provide data and statistics about the attitude of society towards this technology. Then,
I focus on home automation system design, indicating the different possible architectures that a domotic system can have.

Chapter 3 is about voice assistance, another very important part in this project. The objective is similar to the previous 
chapter, I explain what are the virtual assistants and, more precisely, the voice assistance technology. I give examples of where
we can find virtual assistants and, in the second section of this chapter, I indicate the capabilities and services that virtual 
assistants can provide.

In chapter 4, I analyze many different products related to this project. It is divided in three sections: home automation 
systems, home automation devices and voice assistants. In the first section, I explore the most popular home automation systems
on the market, as well as other open source software. In the second one, I explore home automation devices that are made for very
different purposes. I classify them by type and indicate the pros and cons for each one, and regarding their integration with openHAB.
In the third section, I do the same for voice assistants, exploring the main systems currently in the market. For the home automation 
systems and for the voice assistants, I end their sections with a comparative table of all the options I have presented.

Chapter 5 offers a deeper insight into openHAB, a home automation system previously presented in chapter 4. OpenHAB is a 
huge system worth exploring in depth, and in this chapter I introduce it, as well as its history and structure. I then explore its 
main concepts from a \textit{logical} point of view, and next I offer a developer's perspective, explaining the internal organization 
of the software, its installation and other technical concepts.

I explain the entire project development process in chapter 6. First, I provide an analysis of the system from a software
engineering perspective, indicating product specification, system analysis and system design. Then, I describe the implementation
process from the installation of the system. This chapter is mainly technical, and in this section I provide snippets of code that
I have used in the project. Also, the  appendix A is related to this chapter, where i include the full script that composes 
the voice assistant.

This work ends with chapter 7. Here I analyze the obtained results and give ideas for future developments based on this
work. I also analyze the fulfillment of the specific objectives specified in this chapter.


